%%%%%%%%%%%%%%%%%%%%%%%%%%%%%%%%%%%%%%%%%%%%%%
% Cambridge University Press journal codes:
% aas, jps, nws, pla, ram, ehs, nlp
%
% NOTE:
% This draft uses journal=ehs as a valid placeholder.
% Change the journal code above if you target a different CUP journal.
%%%%%%%%%%%%%%%%%%%%%%%%%%%%%%%%%%%%%%%%%%%%%%

\PassOptionsToPackage{
  colorlinks=true,
  pdfborder={0 0 0},
  citecolor=blue,
  linkcolor=blue,
  urlcolor=blue
}{hyperref}

\documentclass[
  journal=ehs,
  manuscript=article,
  year=2026,
  volume=21
]{cup-journal}

\usepackage{amsmath}
\usepackage[nopatch]{microtype}
\usepackage{booktabs}
\usepackage{graphicx}
\usepackage{orcidlink}
\usepackage{threeparttable}
\usepackage{tabularx}
\usepackage{ragged2e}

\title{Epidemiological Semiparametric Machine Learning:\\Retrospective Alcohol and Drug Use Cohort Study}

\author{Vansh Singh Ruhela \orcidlink{0009-0004-1750-3592}}
\affiliation{Collaborative Specialization in Addiction Studies, Dalla Lana School of Public Health, University of Toronto, Toronto, ON, Canada}
\email[Vansh Singh Ruhela]{vansh.ruhela@mail.utoronto.ca}
\alsoaffiliation{School of Graduate Studies, University of Toronto, Toronto, ON, Canada}
\alsoaffiliation{Developer Program, Google LLC, Mountain View, CA, USA}

\addbibresource{13_research_proposal_cup.bib}

\keywords{epidemiology, causal inference, semiparametric machine learning, CPADS, cannabis use, heavy drinking}

\begin{document}

\begin{abstract}
\noindent\textbf{Background:}
Alcohol and cannabis co-use in postsecondary populations is a major public health concern in Canada. The Canadian Postsecondary Education Alcohol and Drug Use Survey (CPADS) 2021--2022 Public Use Microdata File (PUMF) provides a large observational sample suitable for prevalence, association, and causal-effect estimation under explicit assumptions \cite{HealthCanada_CPADSsummary,HealthCanada_CPADS_2021_2022_PUMF_data}.

\medskip
\noindent\textbf{Objective:}
To estimate and interpret the relationship between past-year cannabis use and past-30-day heavy drinking, with transparent handling of sampling weights, confounding adjustment, and causal identification assumptions.

\medskip
\noindent\textbf{Study Design:}
Secondary analysis of respondent-level CPADS records (wrangled \(N=40{,}931\)). The main analyses include survey-weighted prevalence and logistic regression, propensity-score weighting, matching/subclassification, and sensitivity analyses.

\medskip
\noindent\textbf{Preliminary Signal From Current Workflow:}
Weighted heavy-drinking prevalence was \(45.8\%\) (95\% CI \(45.2\%\) to \(46.5\%\)); adjusted survey-logistic odds ratio for cannabis exposure was \(5.71\) (95\% CI \(5.38\) to \(6.07\)); and IPW ATE was \(0.394\) (95\% CI \(0.384\) to \(0.405\)).

\medskip
\noindent\textbf{Knowledge Translation:}
The project is designed for reproducible dissemination through a code-first workflow (R/Quarto/\LaTeX{} + Git), with links to official open-data access rather than redistributing CPADS microdata.
\end{abstract}

\clearpage
\tableofcontents
\clearpage

\section{Research Proposal}

\subsection{Background and Rationale}
This proposal analyzes the CPADS 2021--2022 PUMF as a retrospective observational cohort-style dataset for substance-use epidemiology \cite{HealthCanada_CPADS_2021_2022_PUMF_data,HealthCanada2025_CPADS_technical_notes}. The clinical framing follows official nosologies: DSM-5-TR for Alcohol Use Disorder (AUD) and Cannabis Use Disorder (CUD) \cite{apa2022_dsm5tr}, and ICD-11 Mortality and Morbidity Statistics domains for disorders due to alcohol (6C40) and cannabis (6C41) \cite{who2026_icd11_mms}.

The proposal treats CPADS variables as \emph{population-level behavioral indicators} aligned to diagnostic domains. CPADS is not a clinical diagnostic interview instrument and is therefore used for epidemiologic inference, not individual diagnosis.

\subsection{Objective and Research Questions}
\textbf{Primary objective:} estimate the direction and magnitude of association between cannabis exposure and heavy drinking, then estimate causal risk differences under explicit assumptions.

\textbf{Research questions:}
\begin{enumerate}
  \item What is the survey-weighted prevalence of heavy drinking in the CPADS target population?
  \item What is the adjusted association between cannabis exposure and heavy drinking on the odds scale?
  \item Under identification assumptions, what are ATE/ATT/ATC estimates on the risk-difference scale?
  \item How sensitive are findings to modeling choices (e.g., weighting, matching, oversampling)?
\end{enumerate}

\subsection{Data Source, Sampling, and Analysis Population}
\textbf{Data source:} CPADS 2021--2022 PUMF and technical documentation \cite{HealthCanada_CPADS_2021_2022_PUMF_data,HealthCanada_CPADS_2021_2022_PUMF_user_guide,HealthCanada2025_CPADS_technical_notes}.

\textbf{Sampling and weights:}
\begin{itemize}
  \item Sampling is determined by the original CPADS survey design, not by this analysis.
  \item The analysis uses the provided survey weight variable \texttt{wtpumf} in weighted phases.
  \item Survey design object in code: \texttt{svydesign(ids=\textasciitilde 1, weights=\textasciitilde wtpumf, data=...)} \cite{LumleyGaoSchneider2025_survey,srvyr}.
\end{itemize}

\textbf{Observed analysis counts from the current workflow:}
\begin{itemize}
  \item Cleaned dataset: \(N=40{,}931\).
  \item Non-missing heavy drinking outcome: \(n=31{,}719\).
  \item Survey-logistic complete cases: \(n=30{,}989\).
  \item Causal complete cases (includes physical health): \(n=30{,}890\).
\end{itemize}

\subsection{Variable Construction and Data Operations}
The data-wrangling script applies deterministic recoding rules:
\begin{itemize}
  \item Special codes \(97,98,99,997,998,999\) are recoded to \texttt{NA} for non-preserved columns.
  \item \texttt{gender} from \texttt{dvdemq01}: Woman / Man / Transgender-Non-binary.
  \item \texttt{age\_group} from \texttt{age\_groups}: 16--19, 20--22, 23--25, 26+.
  \item \texttt{province\_region} from \texttt{region}: Atlantic, Quebec, Ontario, Western.
  \item \texttt{cannabis\_any\_use} from \texttt{can05}: 1=yes, 0=no.
  \item \texttt{heavy\_drinking\_30d}: uses \texttt{alc12\_30d\_prev\_total} first; falls back to \texttt{alc12\_30d\_prev}.
\end{itemize}

\subsection{Exposure, Outcome, Randomization Status, and Matching}
\textbf{Treatment (exposure):} \(T_i=\texttt{cannabis\_any\_use}_i\), binary (past-year cannabis use).

\textbf{Outcome:} \(Y_i=\texttt{heavy\_drinking\_30d}_i\), binary (heavy drinking in past 30 days).

\textbf{Randomization:} there is no experimental randomization or treatment assignment. This is an observational design.

\textbf{How randomization is approximated analytically:}
\begin{itemize}
  \item propensity score modeling (\(\Pr[T=1\mid X]\));
  \item inverse probability weighting (IPW);
  \item nearest-neighbor matching (1:1, ATT-focused, logistic distance);
  \item subclassification matching (10 subclasses, ATE-focused).
\end{itemize}

\subsection{Statistical Analysis Plan}
\subsubsection{Survey-weighted descriptive inference}
Heavy-drinking prevalence is estimated with survey-weighted means and 95\% confidence intervals:
\[
\hat{p}=\widehat{E}_w[Y].
\]
Group-specific prevalence is estimated by weighted subgroup means.

\subsubsection{Survey-weighted logistic model}
Primary association model:
\[
\log\!\left(\frac{\Pr(Y_i=1\mid T_i,X_i)}{1-\Pr(Y_i=1\mid T_i,X_i)}\right)
=\beta_0+\beta_1 T_i+\beta_2^\top X_i,
\]
fit with \texttt{svyglm(..., family=quasibinomial())}. The adjusted association is reported as
\[
\text{OR}=\exp(\beta_1).
\]

\subsubsection{Causal estimands and propensity weighting}
Potential-outcomes estimands:
\[
\text{ATE}=E[Y(1)-Y(0)],\quad
\text{ATT}=E[Y(1)-Y(0)\mid T=1],\quad
\text{ATC}=E[Y(1)-Y(0)\mid T=0].
\]
Propensity score:
\[
e(X_i)=\Pr(T_i=1\mid X_i).
\]
ATE IPW:
\[
w_i^{ATE}=\frac{T_i}{e(X_i)}+\frac{1-T_i}{1-e(X_i)}.
\]
Current scripts use trimming to stabilize extreme weights:
\begin{itemize}
  \item one IPW workflow trims final weights at the 1st/99th percentiles;
  \item treatment-effect workflow clips propensity scores to \([0.01,0.99]\) before weighting.
\end{itemize}
Bootstrap uncertainty is computed with \(500\) resamples in IPW and g-computation phases.

\subsubsection{Additional estimators and sensitivity analyses}
\begin{itemize}
  \item G-computation and AIPW (package-based when available, manual fallback otherwise) \cite{aipw}.
  \item Matching/subclassification with \texttt{MatchIt} and balance diagnostics with \texttt{cobalt}/\texttt{WeightIt} \cite{MatchIt,cobalt,WeightIt2026}.
  \item SMOTE/oversampling sensitivity model is fit as an auxiliary analysis (non-survey-weighted) \cite{Chawla2002SMOTE}.
  \item Optional semiparametric DML extensions are planned for robustness checks \cite{Bach2025DMLinR}.
\end{itemize}

\subsection{Interpretation of Key Existing Initial Results}
\begin{table}[htbp]
\centering
\caption{Interpretation of key current estimates from the existing workflow outputs}
\label{tab:key_estimate_interpretation}
\begin{threeparttable}
\begin{tabularx}{\textwidth}{@{}l l X@{}}
\toprule
Estimate & Value & Interpretation \\
\midrule
Heavy drinking prevalence (weighted) &
\(45.8\%\) (95\% CI \(45.2\%\) to \(46.5\%\)) &
After applying CPADS survey weights, the estimated population prevalence is about 46 heavy-drinking students per 100. This is a population prevalence estimate, not a causal effect. \\
Cannabis effect (survey-logistic OR) &
\(\text{OR}=5.71\) (95\% CI \(5.38\) to \(6.07\)) &
Holding modeled covariates fixed, the odds of heavy drinking are estimated to be 5.71 times higher for cannabis users than non-users. This is an adjusted \emph{associational} odds ratio, not a risk ratio and not automatically causal. \\
ATE (IPW) &
\(0.394\) (95\% CI \(0.384\) to \(0.405\)) &
On the risk-difference scale, the model estimates a 39.4 percentage-point higher heavy-drinking probability under cannabis exposure versus no exposure, averaged over the analysis population, \emph{if} causal assumptions hold. \\
\bottomrule
\end{tabularx}
\end{threeparttable}
\end{table}

\subsection{Causal Assumptions and Scope of Inference}
Causal interpretation is conditional on:
\begin{enumerate}
  \item consistency (observed outcome equals potential outcome under observed exposure),
  \item conditional exchangeability given measured covariates,
  \item positivity/overlap (nonzero treatment probability for relevant covariate strata),
  \item correct specification or robustness of nuisance models.
\end{enumerate}

Accordingly, the OR is interpreted as adjusted association, while ATE/ATT/ATC are interpreted as causal targets only under the above assumptions. Residual confounding remains a key threat in any observational design.

\subsection{Known Methodological Caveats to Address in the Final Study}
\begin{itemize}
  \item Weighted and unweighted estimands are mixed across workflow phases and must be labeled explicitly.
  \item CATE outputs in the current script are subgroup contrasts and are marked as unadjusted.
  \item Linear probability approximations are used in some matched-data standard-error calculations.
  \item SMOTE sensitivity models do not preserve the original survey design weighting.
\end{itemize}

\subsection{Knowledge Translation and Reproducibility}
\textbf{Reproducibility products:}
\begin{itemize}
  \item executable R scripts for each analysis phase;
  \item Quarto/R Markdown and \LaTeX{} manuscript sources;
  \item version-controlled outputs and provenance tables;
  \item session metadata (software versions and computational environment).
\end{itemize}

\textbf{Data access policy:}
CPADS microdata are available under the Government of Canada open-data framework; this project distributes code and reproducibility instructions, not duplicate raw microdata files \cite{HealthCanada_CPADS_2021_2022_PUMF_data}.

\paragraph{Funding Statement}
No external industry funding supported this CPADS proposal analysis.

\paragraph{Competing Interests}
Any external funding unrelated to this CPADS project is disclosed separately and did not influence the present design, analysis, interpretation, or reporting.

\paragraph{Ethical Standards}
This is a secondary analysis of de-identified open microdata. No direct participant contact or intervention occurred.

\paragraph{Data Availability Statement}
CPADS 2021--2022 PUMF and documentation are publicly available from the Government of Canada open-data portal \cite{HealthCanada_CPADS_2021_2022_PUMF_data}.

\paragraph{Author Contributions}
Conceptualization, methodology, software, formal analysis, writing, and revision: VSR.

\printbibliography

\end{document}
